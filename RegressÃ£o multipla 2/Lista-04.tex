% Options for packages loaded elsewhere
\PassOptionsToPackage{unicode}{hyperref}
\PassOptionsToPackage{hyphens}{url}
%
\documentclass[
]{article}
\usepackage{lmodern}
\usepackage{amssymb,amsmath}
\usepackage{ifxetex,ifluatex}
\ifnum 0\ifxetex 1\fi\ifluatex 1\fi=0 % if pdftex
  \usepackage[T1]{fontenc}
  \usepackage[utf8]{inputenc}
  \usepackage{textcomp} % provide euro and other symbols
\else % if luatex or xetex
  \usepackage{unicode-math}
  \defaultfontfeatures{Scale=MatchLowercase}
  \defaultfontfeatures[\rmfamily]{Ligatures=TeX,Scale=1}
\fi
% Use upquote if available, for straight quotes in verbatim environments
\IfFileExists{upquote.sty}{\usepackage{upquote}}{}
\IfFileExists{microtype.sty}{% use microtype if available
  \usepackage[]{microtype}
  \UseMicrotypeSet[protrusion]{basicmath} % disable protrusion for tt fonts
}{}
\makeatletter
\@ifundefined{KOMAClassName}{% if non-KOMA class
  \IfFileExists{parskip.sty}{%
    \usepackage{parskip}
  }{% else
    \setlength{\parindent}{0pt}
    \setlength{\parskip}{6pt plus 2pt minus 1pt}}
}{% if KOMA class
  \KOMAoptions{parskip=half}}
\makeatother
\usepackage{xcolor}
\IfFileExists{xurl.sty}{\usepackage{xurl}}{} % add URL line breaks if available
\IfFileExists{bookmark.sty}{\usepackage{bookmark}}{\usepackage{hyperref}}
\hypersetup{
  pdftitle={Lista 04},
  pdfauthor={Matheus Cougias e Klysman Rezende},
  hidelinks,
  pdfcreator={LaTeX via pandoc}}
\urlstyle{same} % disable monospaced font for URLs
\usepackage[margin=1in]{geometry}
\usepackage{color}
\usepackage{fancyvrb}
\newcommand{\VerbBar}{|}
\newcommand{\VERB}{\Verb[commandchars=\\\{\}]}
\DefineVerbatimEnvironment{Highlighting}{Verbatim}{commandchars=\\\{\}}
% Add ',fontsize=\small' for more characters per line
\usepackage{framed}
\definecolor{shadecolor}{RGB}{248,248,248}
\newenvironment{Shaded}{\begin{snugshade}}{\end{snugshade}}
\newcommand{\AlertTok}[1]{\textcolor[rgb]{0.94,0.16,0.16}{#1}}
\newcommand{\AnnotationTok}[1]{\textcolor[rgb]{0.56,0.35,0.01}{\textbf{\textit{#1}}}}
\newcommand{\AttributeTok}[1]{\textcolor[rgb]{0.77,0.63,0.00}{#1}}
\newcommand{\BaseNTok}[1]{\textcolor[rgb]{0.00,0.00,0.81}{#1}}
\newcommand{\BuiltInTok}[1]{#1}
\newcommand{\CharTok}[1]{\textcolor[rgb]{0.31,0.60,0.02}{#1}}
\newcommand{\CommentTok}[1]{\textcolor[rgb]{0.56,0.35,0.01}{\textit{#1}}}
\newcommand{\CommentVarTok}[1]{\textcolor[rgb]{0.56,0.35,0.01}{\textbf{\textit{#1}}}}
\newcommand{\ConstantTok}[1]{\textcolor[rgb]{0.00,0.00,0.00}{#1}}
\newcommand{\ControlFlowTok}[1]{\textcolor[rgb]{0.13,0.29,0.53}{\textbf{#1}}}
\newcommand{\DataTypeTok}[1]{\textcolor[rgb]{0.13,0.29,0.53}{#1}}
\newcommand{\DecValTok}[1]{\textcolor[rgb]{0.00,0.00,0.81}{#1}}
\newcommand{\DocumentationTok}[1]{\textcolor[rgb]{0.56,0.35,0.01}{\textbf{\textit{#1}}}}
\newcommand{\ErrorTok}[1]{\textcolor[rgb]{0.64,0.00,0.00}{\textbf{#1}}}
\newcommand{\ExtensionTok}[1]{#1}
\newcommand{\FloatTok}[1]{\textcolor[rgb]{0.00,0.00,0.81}{#1}}
\newcommand{\FunctionTok}[1]{\textcolor[rgb]{0.00,0.00,0.00}{#1}}
\newcommand{\ImportTok}[1]{#1}
\newcommand{\InformationTok}[1]{\textcolor[rgb]{0.56,0.35,0.01}{\textbf{\textit{#1}}}}
\newcommand{\KeywordTok}[1]{\textcolor[rgb]{0.13,0.29,0.53}{\textbf{#1}}}
\newcommand{\NormalTok}[1]{#1}
\newcommand{\OperatorTok}[1]{\textcolor[rgb]{0.81,0.36,0.00}{\textbf{#1}}}
\newcommand{\OtherTok}[1]{\textcolor[rgb]{0.56,0.35,0.01}{#1}}
\newcommand{\PreprocessorTok}[1]{\textcolor[rgb]{0.56,0.35,0.01}{\textit{#1}}}
\newcommand{\RegionMarkerTok}[1]{#1}
\newcommand{\SpecialCharTok}[1]{\textcolor[rgb]{0.00,0.00,0.00}{#1}}
\newcommand{\SpecialStringTok}[1]{\textcolor[rgb]{0.31,0.60,0.02}{#1}}
\newcommand{\StringTok}[1]{\textcolor[rgb]{0.31,0.60,0.02}{#1}}
\newcommand{\VariableTok}[1]{\textcolor[rgb]{0.00,0.00,0.00}{#1}}
\newcommand{\VerbatimStringTok}[1]{\textcolor[rgb]{0.31,0.60,0.02}{#1}}
\newcommand{\WarningTok}[1]{\textcolor[rgb]{0.56,0.35,0.01}{\textbf{\textit{#1}}}}
\usepackage{graphicx,grffile}
\makeatletter
\def\maxwidth{\ifdim\Gin@nat@width>\linewidth\linewidth\else\Gin@nat@width\fi}
\def\maxheight{\ifdim\Gin@nat@height>\textheight\textheight\else\Gin@nat@height\fi}
\makeatother
% Scale images if necessary, so that they will not overflow the page
% margins by default, and it is still possible to overwrite the defaults
% using explicit options in \includegraphics[width, height, ...]{}
\setkeys{Gin}{width=\maxwidth,height=\maxheight,keepaspectratio}
% Set default figure placement to htbp
\makeatletter
\def\fps@figure{htbp}
\makeatother
\setlength{\emergencystretch}{3em} % prevent overfull lines
\providecommand{\tightlist}{%
  \setlength{\itemsep}{0pt}\setlength{\parskip}{0pt}}
\setcounter{secnumdepth}{-\maxdimen} % remove section numbering

\title{Lista 04}
\author{Matheus Cougias e Klysman Rezende}
\date{27/08/2020}

\begin{document}
\maketitle

\hypertarget{leitura-de-arquivo}{%
\subsection{Leitura de arquivo}\label{leitura-de-arquivo}}

Realiza a leitura dos dados presentes no arquivo
Base\_DEA\_valores\_medios\_2014-2016.csv. A função read.csv2 foi
utilizada devido ao tipo de separador presente no arquivo e algumas das
colunas foram retiradas pois não são de interesse para análise.

\begin{Shaded}
\begin{Highlighting}[]
\NormalTok{  dados <-}\StringTok{ }\KeywordTok{read.csv2}\NormalTok{(}\StringTok{'Base_DEA_valores_medios_2014-2016.csv'}\NormalTok{)}
\NormalTok{  dados <-}\StringTok{ }\NormalTok{dados[}\DecValTok{4}\OperatorTok{:}\DecValTok{11}\NormalTok{]}
\end{Highlighting}
\end{Shaded}

\hypertarget{anuxe1lise-exploratuxf3ria-dos-dados}{%
\subsection{Análise exploratória dos
dados}\label{anuxe1lise-exploratuxf3ria-dos-dados}}

Como início da análise exploratória dos dados, foi feito um histograma
para ter ideia de como funciona a distribuição da variável resposta.
Pode-se perceber uma tendência de distribuição exponencial nos dados,
pois os as colunas do histograma decrescem de tamanho da direção do eixo
X positivo. Uma teoria provável para a próxima etapa será aplicar um
logaritmo na variável PMSO, para tentar corrigir esse comportamento
exponencial da variável.

Para facilitar uma análise das possíveis situações dos dados, foram
anexadas à base de dados colunas que representam o logaritmo das
variáveis preditoras, e também foi feita uma cópia da base de dados,
alterando os valores do PMSO para o logaritmo da mesma. A partir da
análise dos gráficos gerados comparando tanto as situações da variável
resposta quanto as variáveis preditoras em logaritmo, é perceptivel uma
maior linearidade entre os dados quando ambos os lados estão em
logaritmo, exceto na variável rsub, onde a maior linearidade dos dados
está na situação onde a variável preditora está em escala log e PMSO
está na escala original. Por questão de estética, foram deixados somente
os gráficos onde a distribuição se mostrou mais padronizada.

\begin{Shaded}
\begin{Highlighting}[]
\KeywordTok{require}\NormalTok{(packHV)}
\end{Highlighting}
\end{Shaded}

\begin{verbatim}
## Loading required package: packHV
\end{verbatim}

\begin{verbatim}
## Loading required package: survival
\end{verbatim}

\begin{Shaded}
\begin{Highlighting}[]
\KeywordTok{require}\NormalTok{(exploreR)}
\end{Highlighting}
\end{Shaded}

\begin{verbatim}
## Loading required package: exploreR
\end{verbatim}

\begin{Shaded}
\begin{Highlighting}[]
\KeywordTok{require}\NormalTok{(corrplot)}
\end{Highlighting}
\end{Shaded}

\begin{verbatim}
## Loading required package: corrplot
\end{verbatim}

\begin{verbatim}
## corrplot 0.84 loaded
\end{verbatim}

\begin{Shaded}
\begin{Highlighting}[]
\KeywordTok{hist_boxplot}\NormalTok{(dados}\OperatorTok{$}\NormalTok{PMSOaj, }\DataTypeTok{main=}\StringTok{"Histograma"}\NormalTok{, }\DataTypeTok{xlab=}\StringTok{"PMSO"}\NormalTok{, }\DataTypeTok{ylab=}\StringTok{"Frequência", col="}\NormalTok{light green}\StringTok{")}
\StringTok{rug(dados$PMSOaj)}
\end{Highlighting}
\end{Shaded}

\includegraphics{Lista-04_files/figure-latex/analise exploratoria-1.pdf}

\begin{Shaded}
\begin{Highlighting}[]
\NormalTok{corMat <-}\StringTok{ }\KeywordTok{cor}\NormalTok{(dados, }\DataTypeTok{method=}\StringTok{"spearman"}\NormalTok{)}
\KeywordTok{corrplot}\NormalTok{(corMat, }\DataTypeTok{method =} \StringTok{"ellipse"}\NormalTok{, }\DataTypeTok{type=}\StringTok{"upper"}\NormalTok{, }\DataTypeTok{order=}\StringTok{"AOE"}\NormalTok{, }
          \DataTypeTok{diag=}\OtherTok{FALSE}\NormalTok{, }\DataTypeTok{addgrid.col=}\OtherTok{NA}\NormalTok{, }\DataTypeTok{outline=}\OtherTok{TRUE}\NormalTok{)}
\end{Highlighting}
\end{Shaded}

\includegraphics{Lista-04_files/figure-latex/analise exploratoria-2.pdf}

\begin{Shaded}
\begin{Highlighting}[]
\NormalTok{dados}\OperatorTok{$}\NormalTok{logrsub    <-}\StringTok{ }\KeywordTok{log}\NormalTok{(dados}\OperatorTok{$}\NormalTok{rsub}\OperatorTok{+}\DecValTok{1}\NormalTok{)}
\NormalTok{dados}\OperatorTok{$}\NormalTok{logrdist_a    <-}\StringTok{ }\KeywordTok{log}\NormalTok{(dados}\OperatorTok{$}\NormalTok{rdist_a)}
\NormalTok{dados}\OperatorTok{$}\NormalTok{logralta    <-}\StringTok{ }\KeywordTok{log}\NormalTok{(dados}\OperatorTok{$}\NormalTok{ralta}\OperatorTok{+}\DecValTok{1}\NormalTok{)}
\NormalTok{dados}\OperatorTok{$}\NormalTok{logmponderado    <-}\StringTok{ }\KeywordTok{log}\NormalTok{(dados}\OperatorTok{$}\NormalTok{mponderado)}
\NormalTok{dados}\OperatorTok{$}\NormalTok{logcons    <-}\StringTok{ }\KeywordTok{log}\NormalTok{(dados}\OperatorTok{$}\NormalTok{cons)}
\NormalTok{dados}\OperatorTok{$}\NormalTok{logCHIaj    <-}\StringTok{ }\KeywordTok{log}\NormalTok{(dados}\OperatorTok{$}\NormalTok{CHIaj}\OperatorTok{+}\DecValTok{1}\NormalTok{)}
\NormalTok{dados}\OperatorTok{$}\NormalTok{logPNTaj    <-}\StringTok{ }\KeywordTok{log}\NormalTok{(dados}\OperatorTok{$}\NormalTok{PNTaj}\OperatorTok{+}\DecValTok{1}\NormalTok{)}

\NormalTok{dados2 <-}\StringTok{ }\NormalTok{dados[}\DecValTok{2}\OperatorTok{:}\DecValTok{15}\NormalTok{]}
\NormalTok{dados2}\OperatorTok{$}\NormalTok{logPMSO <-}\StringTok{ }\KeywordTok{log}\NormalTok{(dados}\OperatorTok{$}\NormalTok{PMSOaj)}
  

\KeywordTok{plot}\NormalTok{(PMSOaj }\OperatorTok{~}\StringTok{ }\NormalTok{logrsub, }\DataTypeTok{data=}\NormalTok{dados, }\DataTypeTok{pch=}\DecValTok{19}\NormalTok{, }\DataTypeTok{col=}\StringTok{"blue"}\NormalTok{)}
\end{Highlighting}
\end{Shaded}

\includegraphics{Lista-04_files/figure-latex/analise exploratoria-3.pdf}

\begin{Shaded}
\begin{Highlighting}[]
\KeywordTok{plot}\NormalTok{(logPMSO }\OperatorTok{~}\StringTok{ }\NormalTok{logrdist_a, }\DataTypeTok{data=}\NormalTok{dados2, }\DataTypeTok{pch=}\DecValTok{19}\NormalTok{, }\DataTypeTok{col=}\StringTok{"blue"}\NormalTok{)}
\end{Highlighting}
\end{Shaded}

\includegraphics{Lista-04_files/figure-latex/analise exploratoria-4.pdf}

\begin{Shaded}
\begin{Highlighting}[]
\KeywordTok{plot}\NormalTok{(logPMSO }\OperatorTok{~}\StringTok{ }\NormalTok{logralta, }\DataTypeTok{data=}\NormalTok{dados2, }\DataTypeTok{pch=}\DecValTok{19}\NormalTok{, }\DataTypeTok{col=}\StringTok{"blue"}\NormalTok{)}
\end{Highlighting}
\end{Shaded}

\includegraphics{Lista-04_files/figure-latex/analise exploratoria-5.pdf}

\begin{Shaded}
\begin{Highlighting}[]
\KeywordTok{plot}\NormalTok{(logPMSO }\OperatorTok{~}\StringTok{ }\NormalTok{logmponderado, }\DataTypeTok{data=}\NormalTok{dados2, }\DataTypeTok{pch=}\DecValTok{19}\NormalTok{, }\DataTypeTok{col=}\StringTok{"blue"}\NormalTok{)}
\end{Highlighting}
\end{Shaded}

\includegraphics{Lista-04_files/figure-latex/analise exploratoria-6.pdf}

\begin{Shaded}
\begin{Highlighting}[]
\KeywordTok{plot}\NormalTok{(logPMSO }\OperatorTok{~}\StringTok{ }\NormalTok{logcons, }\DataTypeTok{data=}\NormalTok{dados2, }\DataTypeTok{pch=}\DecValTok{19}\NormalTok{, }\DataTypeTok{col=}\StringTok{"blue"}\NormalTok{)}
\end{Highlighting}
\end{Shaded}

\includegraphics{Lista-04_files/figure-latex/analise exploratoria-7.pdf}

\begin{Shaded}
\begin{Highlighting}[]
\KeywordTok{plot}\NormalTok{(logPMSO }\OperatorTok{~}\StringTok{ }\NormalTok{logPNTaj, }\DataTypeTok{data=}\NormalTok{dados2, }\DataTypeTok{pch=}\DecValTok{19}\NormalTok{, }\DataTypeTok{col=}\StringTok{"blue"}\NormalTok{)}
\end{Highlighting}
\end{Shaded}

\includegraphics{Lista-04_files/figure-latex/analise exploratoria-8.pdf}

\begin{Shaded}
\begin{Highlighting}[]
\KeywordTok{plot}\NormalTok{(logPMSO }\OperatorTok{~}\StringTok{ }\NormalTok{logCHIaj, }\DataTypeTok{data=}\NormalTok{dados2, }\DataTypeTok{pch=}\DecValTok{19}\NormalTok{, }\DataTypeTok{col=}\StringTok{"blue"}\NormalTok{)}
\end{Highlighting}
\end{Shaded}

\includegraphics{Lista-04_files/figure-latex/analise exploratoria-9.pdf}

\hypertarget{modelo-de-regressuxe3o-linear-muxfaltipla}{%
\subsection{Modelo de regressão linear
múltipla}\label{modelo-de-regressuxe3o-linear-muxfaltipla}}

Ainda com a ideia de que tanto a base onde a variável resposta está na
base original, quanto na escala log devem ser testadas, foi realizada
uma regrassão linear múltipla, no objetivo de identificar quais
realmente seriam as variáveis que melhor representam os dados originais.
Quanto comparados os resíduos gerados pelas bases, a que o PMSO está em
escala original, os resíduos geram uma distribuição totalmente confusa,
onde existem diversos pontos fora da normal tanto em valores pequenos
quanto em valores mais altos no gráfico.

Já no caso da regressão linear múltipla aplicada sobre a base de dados
com log de PMSO, os resultados se mostraram mais satisfatórios, mantendo
o nível de R² próximo de 0,99 e normalizando um pouco mais os resíduos
gerados e deixando mais padronizada a curva desses resíduos. Dessa
maneira, o modelo a ser utilizado tomará como variáveis: logPMSO, rsub,
rdist\_a, cons, PNTaj, CHIaj, logralta, logmponderado e logcons.

\begin{Shaded}
\begin{Highlighting}[]
\KeywordTok{require}\NormalTok{(car)}
\end{Highlighting}
\end{Shaded}

\begin{verbatim}
## Loading required package: car
\end{verbatim}

\begin{verbatim}
## Loading required package: carData
\end{verbatim}

\begin{Shaded}
\begin{Highlighting}[]
\NormalTok{modelo <-}\StringTok{ }\KeywordTok{lm}\NormalTok{(PMSOaj }\OperatorTok{~}\StringTok{ }\NormalTok{., }\DataTypeTok{data =}\NormalTok{ dados)}
\NormalTok{modelo <-}\StringTok{ }\KeywordTok{step}\NormalTok{(modelo)}
\end{Highlighting}
\end{Shaded}

\begin{verbatim}
## Start:  AIC=1352.27
## PMSOaj ~ rsub + rdist_a + ralta + mponderado + cons + PNTaj + 
##     CHIaj + logrsub + logrdist_a + logralta + logmponderado + 
##     logcons + logCHIaj + logPNTaj
## 
##                 Df  Sum of Sq        RSS    AIC
## - logcons        1 6.4785e+07 1.5831e+11 1350.3
## - logrdist_a     1 3.4280e+08 1.5858e+11 1350.4
## - logCHIaj       1 3.8795e+08 1.5863e+11 1350.4
## - logmponderado  1 4.0415e+08 1.5864e+11 1350.4
## - logPNTaj       1 1.7826e+09 1.6002e+11 1351.0
## - logralta       1 3.1900e+09 1.6143e+11 1351.5
## - logrsub        1 5.0475e+09 1.6329e+11 1352.2
## <none>                        1.5824e+11 1352.3
## - ralta          1 1.0349e+10 1.6859e+11 1354.1
## - rsub           1 1.1008e+10 1.6925e+11 1354.4
## - mponderado     1 1.4737e+10 1.7298e+11 1355.7
## - cons           1 2.3743e+10 1.8198e+11 1358.8
## - CHIaj          1 3.1841e+10 1.9008e+11 1361.5
## - PNTaj          1 6.9621e+10 2.2786e+11 1372.5
## - rdist_a        1 1.6890e+11 3.2714e+11 1394.6
## 
## Step:  AIC=1350.29
## PMSOaj ~ rsub + rdist_a + ralta + mponderado + cons + PNTaj + 
##     CHIaj + logrsub + logrdist_a + logralta + logmponderado + 
##     logCHIaj + logPNTaj
## 
##                 Df  Sum of Sq        RSS    AIC
## - logrdist_a     1 2.7807e+08 1.5858e+11 1348.4
## - logCHIaj       1 3.4726e+08 1.5865e+11 1348.4
## - logmponderado  1 6.7444e+08 1.5898e+11 1348.5
## - logPNTaj       1 1.9810e+09 1.6029e+11 1349.0
## - logralta       1 3.7364e+09 1.6204e+11 1349.7
## - logrsub        1 4.9842e+09 1.6329e+11 1350.2
## <none>                        1.5831e+11 1350.3
## - ralta          1 1.0559e+10 1.6886e+11 1352.2
## - rsub           1 1.0992e+10 1.6930e+11 1352.4
## - mponderado     1 1.7476e+10 1.7578e+11 1354.7
## - CHIaj          1 3.1878e+10 1.9018e+11 1359.5
## - cons           1 3.5840e+10 1.9415e+11 1360.7
## - PNTaj          1 7.0662e+10 2.2897e+11 1370.8
## - rdist_a        1 1.7756e+11 3.3587e+11 1394.2
## 
## Step:  AIC=1348.4
## PMSOaj ~ rsub + rdist_a + ralta + mponderado + cons + PNTaj + 
##     CHIaj + logrsub + logralta + logmponderado + logCHIaj + logPNTaj
## 
##                 Df  Sum of Sq        RSS    AIC
## - logCHIaj       1 4.0693e+08 1.5899e+11 1346.6
## - logPNTaj       1 2.4642e+09 1.6105e+11 1347.3
## - logmponderado  1 2.9044e+09 1.6149e+11 1347.5
## - logralta       1 3.4979e+09 1.6208e+11 1347.7
## <none>                        1.5858e+11 1348.4
## - logrsub        1 5.8211e+09 1.6440e+11 1348.6
## - ralta          1 1.0306e+10 1.6889e+11 1350.2
## - rsub           1 1.1022e+10 1.6961e+11 1350.5
## - mponderado     1 1.7787e+10 1.7637e+11 1352.9
## - CHIaj          1 3.1657e+10 1.9024e+11 1357.5
## - cons           1 3.7188e+10 1.9577e+11 1359.2
## - PNTaj          1 7.5315e+10 2.3390e+11 1370.1
## - rdist_a        1 1.9811e+11 3.5669e+11 1395.8
## 
## Step:  AIC=1346.56
## PMSOaj ~ rsub + rdist_a + ralta + mponderado + cons + PNTaj + 
##     CHIaj + logrsub + logralta + logmponderado + logPNTaj
## 
##                 Df  Sum of Sq        RSS    AIC
## - logPNTaj       1 2.1025e+09 1.6109e+11 1345.4
## - logmponderado  1 2.5059e+09 1.6150e+11 1345.5
## - logralta       1 3.6020e+09 1.6259e+11 1345.9
## <none>                        1.5899e+11 1346.6
## - logrsub        1 5.7024e+09 1.6469e+11 1346.7
## - ralta          1 1.0573e+10 1.6956e+11 1348.5
## - rsub           1 1.1132e+10 1.7012e+11 1348.7
## - mponderado     1 1.7846e+10 1.7684e+11 1351.0
## - CHIaj          1 3.3229e+10 1.9222e+11 1356.1
## - cons           1 3.6807e+10 1.9580e+11 1357.3
## - PNTaj          1 7.4908e+10 2.3390e+11 1368.1
## - rdist_a        1 2.0133e+11 3.6032e+11 1394.5
## 
## Step:  AIC=1345.36
## PMSOaj ~ rsub + rdist_a + ralta + mponderado + cons + PNTaj + 
##     CHIaj + logrsub + logralta + logmponderado
## 
##                 Df  Sum of Sq        RSS    AIC
## - logralta       1 3.7148e+09 1.6481e+11 1344.8
## - logmponderado  1 5.0200e+09 1.6611e+11 1345.2
## <none>                        1.6109e+11 1345.4
## - logrsub        1 6.8963e+09 1.6799e+11 1345.9
## - rsub           1 1.0043e+10 1.7114e+11 1347.0
## - ralta          1 1.1458e+10 1.7255e+11 1347.5
## - mponderado     1 2.0970e+10 1.8206e+11 1350.8
## - CHIaj          1 3.2585e+10 1.9368e+11 1354.6
## - cons           1 3.5598e+10 1.9669e+11 1355.5
## - PNTaj          1 7.9753e+10 2.4085e+11 1367.9
## - rdist_a        1 2.0504e+11 3.6614e+11 1393.4
## 
## Step:  AIC=1344.75
## PMSOaj ~ rsub + rdist_a + ralta + mponderado + cons + PNTaj + 
##     CHIaj + logrsub + logmponderado
## 
##                 Df  Sum of Sq        RSS    AIC
## - logmponderado  1 1.3288e+09 1.6614e+11 1343.2
## - logrsub        1 5.1050e+09 1.6991e+11 1344.6
## <none>                        1.6481e+11 1344.8
## - ralta          1 8.4528e+09 1.7326e+11 1345.8
## - rsub           1 1.2604e+10 1.7741e+11 1347.2
## - mponderado     1 1.8405e+10 1.8321e+11 1349.2
## - cons           1 3.4365e+10 1.9917e+11 1354.3
## - CHIaj          1 3.5170e+10 1.9998e+11 1354.5
## - PNTaj          1 7.6860e+10 2.4167e+11 1366.1
## - rdist_a        1 2.0303e+11 3.6784e+11 1391.7
## 
## Step:  AIC=1343.24
## PMSOaj ~ rsub + rdist_a + ralta + mponderado + cons + PNTaj + 
##     CHIaj + logrsub
## 
##              Df  Sum of Sq        RSS    AIC
## - logrsub     1 3.9686e+09 1.7011e+11 1342.7
## <none>                     1.6614e+11 1343.2
## - ralta       1 9.5182e+09 1.7565e+11 1344.6
## - rsub        1 1.4847e+10 1.8098e+11 1346.5
## - mponderado  1 1.7170e+10 1.8331e+11 1347.2
## - CHIaj       1 3.3878e+10 2.0001e+11 1352.6
## - cons        1 3.5121e+10 2.0126e+11 1352.9
## - PNTaj       1 7.5731e+10 2.4187e+11 1364.2
## - rdist_a     1 2.0492e+11 3.7105e+11 1390.2
## 
## Step:  AIC=1342.68
## PMSOaj ~ rsub + rdist_a + ralta + mponderado + cons + PNTaj + 
##     CHIaj
## 
##              Df  Sum of Sq        RSS    AIC
## <none>                     1.7011e+11 1342.7
## - ralta       1 7.5557e+09 1.7766e+11 1343.3
## - rsub        1 2.2679e+10 1.9278e+11 1348.3
## - mponderado  1 2.3763e+10 1.9387e+11 1348.7
## - cons        1 3.2150e+10 2.0225e+11 1351.2
## - CHIaj       1 3.4872e+10 2.0498e+11 1352.0
## - PNTaj       1 7.5636e+10 2.4574e+11 1363.1
## - rdist_a     1 2.0415e+11 3.7425e+11 1388.8
\end{verbatim}

\begin{Shaded}
\begin{Highlighting}[]
\KeywordTok{summary}\NormalTok{(modelo)}
\end{Highlighting}
\end{Shaded}

\begin{verbatim}
## 
## Call:
## lm(formula = PMSOaj ~ rsub + rdist_a + ralta + mponderado + cons + 
##     PNTaj + CHIaj, data = dados)
## 
## Residuals:
##     Min      1Q  Median      3Q     Max 
## -204995   -9505   -1396   12785  168409 
## 
## Coefficients:
##               Estimate Std. Error t value Pr(>|t|)    
## (Intercept)  5.615e+03  9.285e+03   0.605  0.54790    
## rsub         6.736e+01  2.534e+01   2.658  0.01036 *  
## rdist_a      2.499e+00  3.134e-01   7.975 1.23e-10 ***
## ralta       -1.215e+01  7.918e+00  -1.534  0.13090    
## mponderado   2.186e-02  8.034e-03   2.721  0.00879 ** 
## cons         7.496e-02  2.368e-02   3.165  0.00257 ** 
## PNTaj        9.490e-02  1.955e-02   4.854 1.11e-05 ***
## CHIaj        2.178e-03  6.607e-04   3.296  0.00175 ** 
## ---
## Signif. codes:  0 '***' 0.001 '**' 0.01 '*' 0.05 '.' 0.1 ' ' 1
## 
## Residual standard error: 56650 on 53 degrees of freedom
## Multiple R-squared:  0.9855, Adjusted R-squared:  0.9836 
## F-statistic: 514.3 on 7 and 53 DF,  p-value: < 2.2e-16
\end{verbatim}

\begin{Shaded}
\begin{Highlighting}[]
\KeywordTok{plot}\NormalTok{(modelo)}
\end{Highlighting}
\end{Shaded}

\includegraphics{Lista-04_files/figure-latex/RLM-1.pdf}
\includegraphics{Lista-04_files/figure-latex/RLM-2.pdf}
\includegraphics{Lista-04_files/figure-latex/RLM-3.pdf}
\includegraphics{Lista-04_files/figure-latex/RLM-4.pdf}

\begin{Shaded}
\begin{Highlighting}[]
\NormalTok{modelo <-}\StringTok{ }\KeywordTok{lm}\NormalTok{(logPMSO }\OperatorTok{~}\StringTok{ }\NormalTok{., }\DataTypeTok{data =}\NormalTok{ dados2)}
\NormalTok{modelo <-}\StringTok{ }\KeywordTok{step}\NormalTok{(modelo)}
\end{Highlighting}
\end{Shaded}

\begin{verbatim}
## Start:  AIC=-148.12
## logPMSO ~ rsub + rdist_a + ralta + mponderado + cons + PNTaj + 
##     CHIaj + logrsub + logrdist_a + logralta + logmponderado + 
##     logcons + logCHIaj + logPNTaj
## 
##                 Df Sum of Sq    RSS     AIC
## - logPNTaj       1   0.00001 3.2899 -150.12
## - mponderado     1   0.00089 3.2907 -150.10
## - logrdist_a     1   0.01054 3.3004 -149.93
## - logrsub        1   0.02909 3.3189 -149.59
## - ralta          1   0.04039 3.3302 -149.38
## - logCHIaj       1   0.06166 3.3515 -148.99
## - cons           1   0.06538 3.3552 -148.92
## <none>                       3.2899 -148.12
## - rsub           1   0.16310 3.4530 -147.17
## - CHIaj          1   0.22732 3.5172 -146.05
## - rdist_a        1   0.28174 3.5716 -145.11
## - logralta       1   0.28344 3.5733 -145.08
## - logmponderado  1   0.32607 3.6159 -144.36
## - logcons        1   0.37571 3.6656 -143.53
## - PNTaj          1   0.40916 3.6990 -142.97
## 
## Step:  AIC=-150.12
## logPMSO ~ rsub + rdist_a + ralta + mponderado + cons + PNTaj + 
##     CHIaj + logrsub + logrdist_a + logralta + logmponderado + 
##     logcons + logCHIaj
## 
##                 Df Sum of Sq    RSS     AIC
## - mponderado     1   0.00099 3.2909 -152.10
## - logrdist_a     1   0.01169 3.3016 -151.91
## - logrsub        1   0.02941 3.3193 -151.58
## - ralta          1   0.04066 3.3305 -151.37
## - logCHIaj       1   0.06525 3.3551 -150.92
## - cons           1   0.06789 3.3578 -150.88
## <none>                       3.2899 -150.12
## - rsub           1   0.16576 3.4556 -149.12
## - CHIaj          1   0.22730 3.5172 -148.05
## - logralta       1   0.28344 3.5733 -147.08
## - rdist_a        1   0.29275 3.5826 -146.92
## - logmponderado  1   0.33520 3.6251 -146.20
## - logcons        1   0.38846 3.6783 -145.31
## - PNTaj          1   0.53085 3.8207 -143.00
## 
## Step:  AIC=-152.1
## logPMSO ~ rsub + rdist_a + ralta + cons + PNTaj + CHIaj + logrsub + 
##     logrdist_a + logralta + logmponderado + logcons + logCHIaj
## 
##                 Df Sum of Sq    RSS     AIC
## - logrdist_a     1   0.01114 3.3020 -153.90
## - logrsub        1   0.03016 3.3210 -153.55
## - ralta          1   0.04002 3.3309 -153.37
## - logCHIaj       1   0.06426 3.3551 -152.92
## <none>                       3.2909 -152.10
## - rsub           1   0.17847 3.4693 -150.88
## - CHIaj          1   0.24514 3.5360 -149.72
## - cons           1   0.26090 3.5518 -149.45
## - logralta       1   0.28442 3.5753 -149.05
## - rdist_a        1   0.31011 3.6010 -148.61
## - logmponderado  1   0.41567 3.7065 -146.85
## - logcons        1   0.49797 3.7888 -145.51
## - PNTaj          1   0.53023 3.8211 -144.99
## 
## Step:  AIC=-153.9
## logPMSO ~ rsub + rdist_a + ralta + cons + PNTaj + CHIaj + logrsub + 
##     logralta + logmponderado + logcons + logCHIaj
## 
##                 Df Sum of Sq    RSS     AIC
## - logrsub        1   0.03868 3.3407 -155.19
## - ralta          1   0.04198 3.3440 -155.13
## - logCHIaj       1   0.06137 3.3634 -154.77
## <none>                       3.3020 -153.90
## - rsub           1   0.18580 3.4878 -152.56
## - CHIaj          1   0.24735 3.5493 -151.49
## - logralta       1   0.28659 3.5886 -150.82
## - cons           1   0.35643 3.6584 -149.64
## - rdist_a        1   0.41427 3.7163 -148.69
## - logmponderado  1   0.42108 3.7231 -148.58
## - PNTaj          1   0.52063 3.8226 -146.97
## - logcons        1   0.68023 3.9822 -144.47
## 
## Step:  AIC=-155.19
## logPMSO ~ rsub + rdist_a + ralta + cons + PNTaj + CHIaj + logralta + 
##     logmponderado + logcons + logCHIaj
## 
##                 Df Sum of Sq    RSS     AIC
## - logCHIaj       1   0.05079 3.3915 -156.27
## - ralta          1   0.05823 3.3989 -156.13
## <none>                       3.3407 -155.19
## - rsub           1   0.14735 3.4880 -154.55
## - CHIaj          1   0.24838 3.5891 -152.81
## - logralta       1   0.32400 3.6647 -151.54
## - cons           1   0.34936 3.6900 -151.12
## - logmponderado  1   0.38398 3.7247 -150.55
## - rdist_a        1   0.41969 3.7604 -149.97
## - PNTaj          1   0.56451 3.9052 -147.66
## - logcons        1   0.79225 4.1329 -144.21
## 
## Step:  AIC=-156.27
## logPMSO ~ rsub + rdist_a + ralta + cons + PNTaj + CHIaj + logralta + 
##     logmponderado + logcons
## 
##                 Df Sum of Sq    RSS     AIC
## - ralta          1   0.05089 3.4424 -157.36
## <none>                       3.3915 -156.27
## - rsub           1   0.15559 3.5471 -155.53
## - CHIaj          1   0.21674 3.6082 -154.49
## - logralta       1   0.29781 3.6893 -153.13
## - cons           1   0.32685 3.7183 -152.65
## - logmponderado  1   0.33507 3.7265 -152.52
## - rdist_a        1   0.39940 3.7909 -151.47
## - PNTaj          1   0.54665 3.9381 -149.15
## - logcons        1   0.87370 4.2652 -144.28
## 
## Step:  AIC=-157.36
## logPMSO ~ rsub + rdist_a + cons + PNTaj + CHIaj + logralta + 
##     logmponderado + logcons
## 
##                 Df Sum of Sq    RSS     AIC
## <none>                       3.4424 -157.36
## - rsub           1   0.20275 3.6451 -155.87
## - logralta       1   0.24878 3.6911 -155.10
## - logmponderado  1   0.31503 3.7574 -154.02
## - CHIaj          1   0.31598 3.7583 -154.00
## - rdist_a        1   0.45141 3.8938 -151.84
## - cons           1   0.48800 3.9304 -151.27
## - PNTaj          1   0.61626 4.0586 -149.31
## - logcons        1   0.95919 4.4016 -144.36
\end{verbatim}

\begin{Shaded}
\begin{Highlighting}[]
\KeywordTok{summary}\NormalTok{(modelo)}
\end{Highlighting}
\end{Shaded}

\begin{verbatim}
## 
## Call:
## lm(formula = logPMSO ~ rsub + rdist_a + cons + PNTaj + CHIaj + 
##     logralta + logmponderado + logcons, data = dados2)
## 
## Residuals:
##     Min      1Q  Median      3Q     Max 
## -0.7326 -0.1455 -0.0058  0.1402  0.7516 
## 
## Coefficients:
##                 Estimate Std. Error t value Pr(>|t|)    
## (Intercept)    5.863e-01  5.336e-01   1.099 0.276973    
## rsub           1.831e-04  1.046e-04   1.750 0.086009 .  
## rdist_a        2.579e-06  9.875e-07   2.611 0.011759 *  
## cons          -1.696e-07  6.248e-08  -2.715 0.008971 ** 
## PNTaj          2.838e-07  9.302e-08   3.051 0.003585 ** 
## CHIaj          6.258e-09  2.865e-09   2.185 0.033438 *  
## logralta       4.978e-02  2.568e-02   1.939 0.057989 .  
## logmponderado  2.944e-01  1.349e-01   2.181 0.033693 *  
## logcons        5.208e-01  1.368e-01   3.806 0.000373 ***
## ---
## Signif. codes:  0 '***' 0.001 '**' 0.01 '*' 0.05 '.' 0.1 ' ' 1
## 
## Residual standard error: 0.2573 on 52 degrees of freedom
## Multiple R-squared:  0.983,  Adjusted R-squared:  0.9804 
## F-statistic: 376.9 on 8 and 52 DF,  p-value: < 2.2e-16
\end{verbatim}

\begin{Shaded}
\begin{Highlighting}[]
\KeywordTok{plot}\NormalTok{(modelo)}
\end{Highlighting}
\end{Shaded}

\includegraphics{Lista-04_files/figure-latex/RLM-5.pdf}
\includegraphics{Lista-04_files/figure-latex/RLM-6.pdf}
\includegraphics{Lista-04_files/figure-latex/RLM-7.pdf}
\includegraphics{Lista-04_files/figure-latex/RLM-8.pdf}

\hypertarget{predizendo-o-pmso-utilizando-a-base-de-dados-original}{%
\subsection{Predizendo o PMSO utilizando a base de dados
original}\label{predizendo-o-pmso-utilizando-a-base-de-dados-original}}

Utilizando a base de dados original, o R² preditivo encontrado foi de
0.97, podendo ser considerada como uma boa base para predizer os valores
do ano de 2017.

Pela análise da árvore para essa base de dados, o valor do R² preditivo
decresce para 0.6276, independente para o valor utilizado como maxdepth,
mostrando que o modelo não consegue prever corretamente dados futuros
para a variável PMSO.

\begin{Shaded}
\begin{Highlighting}[]
\KeywordTok{require}\NormalTok{(rpart)}
\end{Highlighting}
\end{Shaded}

\begin{verbatim}
## Loading required package: rpart
\end{verbatim}

\begin{Shaded}
\begin{Highlighting}[]
\KeywordTok{require}\NormalTok{(rpart.plot)}
\end{Highlighting}
\end{Shaded}

\begin{verbatim}
## Loading required package: rpart.plot
\end{verbatim}

\begin{Shaded}
\begin{Highlighting}[]
\NormalTok{dados <-}\StringTok{ }\NormalTok{dados[}\DecValTok{1}\OperatorTok{:}\DecValTok{8}\NormalTok{]}
\NormalTok{y    <-}\StringTok{ }\NormalTok{dados}\OperatorTok{$}\NormalTok{PMSOaj}
\NormalTok{yhat <-}\StringTok{ }\KeywordTok{rep}\NormalTok{(}\OtherTok{NA}\NormalTok{, }\KeywordTok{nrow}\NormalTok{(dados))}

\ControlFlowTok{for}\NormalTok{(cont }\ControlFlowTok{in} \DecValTok{1}\OperatorTok{:}\KeywordTok{nrow}\NormalTok{(dados))\{}
\NormalTok{   modelo <-}\StringTok{ }\KeywordTok{lm}\NormalTok{(PMSOaj }\OperatorTok{~}\StringTok{ }\NormalTok{., }\DataTypeTok{data=}\NormalTok{dados[}\OperatorTok{-}\NormalTok{cont,])}
   
\NormalTok{   yhat[cont] <-}\StringTok{ }\KeywordTok{predict}\NormalTok{(modelo, }\DataTypeTok{newdata=}\NormalTok{dados[cont,])}
\NormalTok{\}}

\KeywordTok{plot}\NormalTok{(yhat }\OperatorTok{~}\StringTok{ }\NormalTok{y, }\DataTypeTok{pch=}\DecValTok{19}\NormalTok{, }\DataTypeTok{col=}\StringTok{"blue"}\NormalTok{)}
\KeywordTok{abline}\NormalTok{(}\DataTypeTok{a=}\DecValTok{0}\NormalTok{, }\DataTypeTok{b=}\DecValTok{1}\NormalTok{, }\DataTypeTok{lwd=}\DecValTok{2}\NormalTok{, }\DataTypeTok{col=}\StringTok{"red"}\NormalTok{)}
\end{Highlighting}
\end{Shaded}

\includegraphics{Lista-04_files/figure-latex/predizendo-1.pdf}

\begin{Shaded}
\begin{Highlighting}[]
\NormalTok{R2pred <-}\StringTok{ }\DecValTok{1} \OperatorTok{-}\StringTok{ }\KeywordTok{sum}\NormalTok{( (y}\OperatorTok{-}\NormalTok{yhat)}\OperatorTok{^}\DecValTok{2}\NormalTok{ )}\OperatorTok{/}\KeywordTok{sum}\NormalTok{( (y}\OperatorTok{-}\KeywordTok{mean}\NormalTok{(y))}\OperatorTok{^}\DecValTok{2}\NormalTok{ )}
\KeywordTok{print}\NormalTok{(R2pred)}
\end{Highlighting}
\end{Shaded}

\begin{verbatim}
## [1] 0.9700012
\end{verbatim}

\begin{Shaded}
\begin{Highlighting}[]
\NormalTok{modelo <-}\StringTok{ }\KeywordTok{rpart}\NormalTok{(PMSOaj }\OperatorTok{~}\StringTok{  }\NormalTok{rsub }\OperatorTok{+}\StringTok{ }\NormalTok{rdist_a }\OperatorTok{+}\StringTok{ }\NormalTok{ralta }\OperatorTok{+}\StringTok{ }\NormalTok{mponderado }\OperatorTok{+}\StringTok{ }\NormalTok{cons }\OperatorTok{+}\StringTok{ }\NormalTok{PNTaj }\OperatorTok{+}\StringTok{ }\NormalTok{CHIaj,  }\DataTypeTok{data=}\NormalTok{dados)}
 
\KeywordTok{rpart.plot}\NormalTok{(modelo)}
\end{Highlighting}
\end{Shaded}

\includegraphics{Lista-04_files/figure-latex/predizendo-2.pdf}

\begin{Shaded}
\begin{Highlighting}[]
\NormalTok{y    <-}\StringTok{ }\NormalTok{dados}\OperatorTok{$}\NormalTok{PMSOaj}
\NormalTok{yhat <-}\StringTok{ }\KeywordTok{rep}\NormalTok{(}\OtherTok{NA}\NormalTok{, }\KeywordTok{nrow}\NormalTok{(dados))}

\ControlFlowTok{for}\NormalTok{(cont }\ControlFlowTok{in} \DecValTok{1}\OperatorTok{:}\KeywordTok{nrow}\NormalTok{(dados))\{}
\NormalTok{ modelo <-}\StringTok{ }\KeywordTok{rpart}\NormalTok{(PMSOaj }\OperatorTok{~}\StringTok{  }\NormalTok{rsub }\OperatorTok{+}\StringTok{ }\NormalTok{rdist_a }\OperatorTok{+}\StringTok{ }\NormalTok{ralta }\OperatorTok{+}\StringTok{ }\NormalTok{mponderado }\OperatorTok{+}\StringTok{ }\NormalTok{cons }\OperatorTok{+}\StringTok{ }\NormalTok{PNTaj }\OperatorTok{+}\StringTok{ }\NormalTok{CHIaj, }\DataTypeTok{data=}\NormalTok{dados[}\OperatorTok{-}\NormalTok{cont,],}
            \DataTypeTok{control =} \KeywordTok{rpart.control}\NormalTok{(}\DataTypeTok{maxdepth=}\DecValTok{2}\NormalTok{))}
   
\NormalTok{ yhat[cont] <-}\StringTok{ }\KeywordTok{predict}\NormalTok{(modelo, }\DataTypeTok{newdata=}\NormalTok{dados[cont,])}
\NormalTok{\}}
 
\KeywordTok{plot}\NormalTok{(yhat }\OperatorTok{~}\StringTok{ }\NormalTok{y, }\DataTypeTok{pch=}\DecValTok{19}\NormalTok{, }\DataTypeTok{col=}\StringTok{"blue"}\NormalTok{)}
\KeywordTok{abline}\NormalTok{(}\DataTypeTok{a=}\DecValTok{0}\NormalTok{, }\DataTypeTok{b=}\DecValTok{1}\NormalTok{, }\DataTypeTok{lwd=}\DecValTok{2}\NormalTok{, }\DataTypeTok{col=}\StringTok{"red"}\NormalTok{)}
\end{Highlighting}
\end{Shaded}

\includegraphics{Lista-04_files/figure-latex/predizendo-3.pdf}

\begin{Shaded}
\begin{Highlighting}[]
\NormalTok{R2pred <-}\StringTok{ }\DecValTok{1} \OperatorTok{-}\StringTok{ }\KeywordTok{sum}\NormalTok{( (y}\OperatorTok{-}\NormalTok{yhat)}\OperatorTok{^}\DecValTok{2}\NormalTok{ )}\OperatorTok{/}\KeywordTok{sum}\NormalTok{( (y}\OperatorTok{-}\KeywordTok{mean}\NormalTok{(y))}\OperatorTok{^}\DecValTok{2}\NormalTok{ )}
\KeywordTok{print}\NormalTok{(R2pred)}
\end{Highlighting}
\end{Shaded}

\begin{verbatim}
## [1] 0.6276422
\end{verbatim}

\hypertarget{predizendo-o-pmso-utilizando-a-base-de-dados-do-modelo-mais-factuxedvel}{%
\subsection{Predizendo o PMSO utilizando a base de dados do modelo mais
factível}\label{predizendo-o-pmso-utilizando-a-base-de-dados-do-modelo-mais-factuxedvel}}

Pelo fato do R² preditivo da base original já ser extremamente alto, um
pequeno acréscimo pode ser considerado como um ganho. Com o modelo
encontrado na primeira parte do trabalho, o R² preditivo gerado foi de
0.9763, melhor que quando comparado ao resultado da base original.
Aliado a esse melhor R² preditivo, o ganho também se dá com o
comportamento mais padronizado dos resíduos desse segundo modelo, já que
o comportamento dos resíduos originalmente gera uma solução não factivel
para o problema.

O resultado encontrado com a aplicação da árvore também é mais
satisfatório que na utilização da base original de dados. O R² preditivo
tem um acréscimo para 0.9089, prevendo melhor os valores que a base
original.

\begin{Shaded}
\begin{Highlighting}[]
\NormalTok{y    <-}\StringTok{ }\NormalTok{dados2}\OperatorTok{$}\NormalTok{logPMSO}
\NormalTok{yhat <-}\StringTok{ }\KeywordTok{rep}\NormalTok{(}\OtherTok{NA}\NormalTok{, }\KeywordTok{nrow}\NormalTok{(dados))}

\ControlFlowTok{for}\NormalTok{(cont }\ControlFlowTok{in} \DecValTok{1}\OperatorTok{:}\KeywordTok{nrow}\NormalTok{(dados2))\{}
\NormalTok{   modelo <-}\StringTok{ }\KeywordTok{lm}\NormalTok{(logPMSO }\OperatorTok{~}\StringTok{ }\NormalTok{rsub }\OperatorTok{+}\StringTok{ }\NormalTok{rdist_a }\OperatorTok{+}\StringTok{ }\NormalTok{cons }\OperatorTok{+}\StringTok{ }\NormalTok{PNTaj }\OperatorTok{+}\StringTok{ }\NormalTok{CHIaj }\OperatorTok{+}\StringTok{ }\NormalTok{logralta }\OperatorTok{+}\StringTok{ }\NormalTok{logmponderado }\OperatorTok{+}\StringTok{ }\NormalTok{logcons, }\DataTypeTok{data=}\NormalTok{dados2[}\OperatorTok{-}\NormalTok{cont,])}
   
\NormalTok{   yhat[cont] <-}\StringTok{ }\KeywordTok{predict}\NormalTok{(modelo, }\DataTypeTok{newdata=}\NormalTok{dados2[cont,])}
\NormalTok{\}}

\KeywordTok{plot}\NormalTok{(yhat }\OperatorTok{~}\StringTok{ }\NormalTok{y, }\DataTypeTok{pch=}\DecValTok{19}\NormalTok{, }\DataTypeTok{col=}\StringTok{"blue"}\NormalTok{)}
\KeywordTok{abline}\NormalTok{(}\DataTypeTok{a=}\DecValTok{0}\NormalTok{, }\DataTypeTok{b=}\DecValTok{1}\NormalTok{, }\DataTypeTok{lwd=}\DecValTok{2}\NormalTok{, }\DataTypeTok{col=}\StringTok{"red"}\NormalTok{)}
\end{Highlighting}
\end{Shaded}

\includegraphics{Lista-04_files/figure-latex/unnamed-chunk-1-1.pdf}

\begin{Shaded}
\begin{Highlighting}[]
\NormalTok{R2pred <-}\StringTok{ }\DecValTok{1} \OperatorTok{-}\StringTok{ }\KeywordTok{sum}\NormalTok{( (y}\OperatorTok{-}\NormalTok{yhat)}\OperatorTok{^}\DecValTok{2}\NormalTok{ )}\OperatorTok{/}\KeywordTok{sum}\NormalTok{( (y}\OperatorTok{-}\KeywordTok{mean}\NormalTok{(y))}\OperatorTok{^}\DecValTok{2}\NormalTok{ )}
\KeywordTok{print}\NormalTok{(R2pred)}
\end{Highlighting}
\end{Shaded}

\begin{verbatim}
## [1] 0.9763237
\end{verbatim}

\begin{Shaded}
\begin{Highlighting}[]
\NormalTok{modelo <-}\StringTok{ }\KeywordTok{rpart}\NormalTok{(logPMSO }\OperatorTok{~}\StringTok{ }\NormalTok{rsub }\OperatorTok{+}\StringTok{ }\NormalTok{rdist_a }\OperatorTok{+}\StringTok{ }\NormalTok{cons }\OperatorTok{+}\StringTok{ }\NormalTok{PNTaj }\OperatorTok{+}\StringTok{ }\NormalTok{CHIaj }\OperatorTok{+}\StringTok{ }\NormalTok{logralta }\OperatorTok{+}\StringTok{ }\NormalTok{logmponderado }\OperatorTok{+}\StringTok{ }\NormalTok{logcons, }\DataTypeTok{data=}\NormalTok{dados2)}
 
\KeywordTok{rpart.plot}\NormalTok{(modelo)}
\end{Highlighting}
\end{Shaded}

\includegraphics{Lista-04_files/figure-latex/unnamed-chunk-1-2.pdf}

\begin{Shaded}
\begin{Highlighting}[]
\NormalTok{y    <-}\StringTok{ }\NormalTok{dados2}\OperatorTok{$}\NormalTok{logPMSO}
\NormalTok{yhat <-}\StringTok{ }\KeywordTok{rep}\NormalTok{(}\OtherTok{NA}\NormalTok{, }\KeywordTok{nrow}\NormalTok{(dados2))}

\ControlFlowTok{for}\NormalTok{(cont }\ControlFlowTok{in} \DecValTok{1}\OperatorTok{:}\KeywordTok{nrow}\NormalTok{(dados2))\{}
\NormalTok{ modelo <-}\StringTok{ }\KeywordTok{rpart}\NormalTok{(logPMSO }\OperatorTok{~}\StringTok{ }\NormalTok{rsub }\OperatorTok{+}\StringTok{ }\NormalTok{rdist_a }\OperatorTok{+}\StringTok{ }\NormalTok{cons }\OperatorTok{+}\StringTok{ }\NormalTok{PNTaj }\OperatorTok{+}\StringTok{ }\NormalTok{CHIaj }\OperatorTok{+}\StringTok{ }\NormalTok{logralta }\OperatorTok{+}\StringTok{ }\NormalTok{logmponderado }\OperatorTok{+}\StringTok{ }\NormalTok{logcons, }\DataTypeTok{data=}\NormalTok{dados2[}\OperatorTok{-}\NormalTok{cont,],}
            \DataTypeTok{control =} \KeywordTok{rpart.control}\NormalTok{(}\DataTypeTok{maxdepth=}\DecValTok{3}\NormalTok{))}
   
\NormalTok{ yhat[cont] <-}\StringTok{ }\KeywordTok{predict}\NormalTok{(modelo, }\DataTypeTok{newdata=}\NormalTok{dados2[cont,])}
\NormalTok{\}}
 
\KeywordTok{plot}\NormalTok{(yhat }\OperatorTok{~}\StringTok{ }\NormalTok{y, }\DataTypeTok{pch=}\DecValTok{19}\NormalTok{, }\DataTypeTok{col=}\StringTok{"blue"}\NormalTok{)}
\KeywordTok{abline}\NormalTok{(}\DataTypeTok{a=}\DecValTok{0}\NormalTok{, }\DataTypeTok{b=}\DecValTok{1}\NormalTok{, }\DataTypeTok{lwd=}\DecValTok{2}\NormalTok{, }\DataTypeTok{col=}\StringTok{"red"}\NormalTok{)}
\end{Highlighting}
\end{Shaded}

\includegraphics{Lista-04_files/figure-latex/unnamed-chunk-1-3.pdf}

\begin{Shaded}
\begin{Highlighting}[]
\NormalTok{R2pred <-}\StringTok{ }\DecValTok{1} \OperatorTok{-}\StringTok{ }\KeywordTok{sum}\NormalTok{( (y}\OperatorTok{-}\NormalTok{yhat)}\OperatorTok{^}\DecValTok{2}\NormalTok{ )}\OperatorTok{/}\KeywordTok{sum}\NormalTok{( (y}\OperatorTok{-}\KeywordTok{mean}\NormalTok{(y))}\OperatorTok{^}\DecValTok{2}\NormalTok{ )}
\KeywordTok{print}\NormalTok{(R2pred)}
\end{Highlighting}
\end{Shaded}

\begin{verbatim}
## [1] 0.9089139
\end{verbatim}

\end{document}
